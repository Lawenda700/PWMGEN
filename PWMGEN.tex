\documentclass{article}
\usepackage{graphicx} 
\title{Documentație PWMGEN}
\author{Brăduț-Mihai Iordache(Lawenda700)\\ Eduard-Alexandru Simion(eddy-24)\\Ovidiu-Mario Timoc(MarioTimoc21)} 

\renewcommand{\contentsname}{Cuprins}
\begin{document}
\maketitle
\newpage
\tableofcontents
\pagestyle{myheadings}
\markright{Brăduț-Mihai Iordache(Lawenda700), Eduard Simion(eddy-24), Mario Timoc(MarioTimoc21)}
\newpage
\section{SPI BRIDGE}
Modulul SPI bridge folosește 2 blocuri \texttt{always}, unul pentru frontul crescător al ceasului 
și celălalt pentru frontul descrescător. Ambele verifică activarea funcției de reset și, în caz că aceasta este activată, resetează valorile output-urilor la valorile default + valoarea celor 2 countere interne.\\
\\
Pe frontul crescător (blocul \texttt{always posedge}) se realizează recepția datelor de la Master. Se citește bit cu bit din linia \texttt{miso} în registrul de shift intern, de la MSB la LSB. Odată ce se citește valoarea ultimului bit (LSB), counter-ul se resetează, iar octetul complet format este salvat în \texttt{data\_latch} și se inversează starea semnalului intern \texttt{flag\_toggle}.\\
\\
Pe frontul descrescător (blocul \texttt{always negedge}) se realizează transmisia datelor către Master. Se scrie pe linia \texttt{mosi} (Output din Slave în această configurație), shiftând biții din \texttt{data\_out} de la MSB la LSB.\\
\\
Sincronizarea cu domeniul de ceas al sistemului (\texttt{clk}) se face printr-un bloc separat care urmărește schimbarea semnalului \texttt{flag\_toggle}. Când se detectează o tranziție, se generează pulsul \texttt{byte\_sync} și se actualizează \texttt{data\_in}. Citirea și scrierea se realizează doar atunci când chip select-ul \texttt{cs\_n} este setat.

\newpage
\section{INSTRUCTION DECODER}
Instruction Decoder-ul funcționează asemenea unui AFD cu 3 stări. Trecerea dintr-o stare în alta se realizează pe frontul crescător al ceasului, iar în caz de reset se revine în starea S0. Stările automatului: \\
\\
S0: Se verifică primirea semnalului \texttt{byte\_sync} care semnalează primirea datelor de input. Se scrie în \texttt{data\_write} pe poziția 0 valoarea bitului de pe poziția 6 din \texttt{data\_write} (ne ajută la scrierea în MSB/LSB din registri), se preia valoarea bitului de pe poziția 7 și, în caz că este 1, \texttt{write} este setat ca 1 și se trece în starea S1 a automatului, iar în caz că este 0, \texttt{read} este setat ca 1 și se trece în starea S2 a automatului. Totodată, biții [5:0] din \texttt{data\_in} sunt scriși în \texttt{addr}.  
În caz că \texttt{byte\_sync} este 0, nu se întâmplă nimic.\\
\\
S1: Se verifică primirea semnalului \texttt{byte\_sync} care semnalează primirea datelor de input.
Datele din \texttt{data\_in} sunt transferate în totalitate în \texttt{data\_write} și se trece în starea S0 a automatului.\\
\\
În caz că \texttt{byte\_sync} este 0, nu se întâmplă nimic.\\
\\
S2: Se trec datele din \texttt{data\_out} în \texttt{data\_read} și se trece în starea S0 a automatului.


\newpage
\section{REGISTERS}
Se stochează adresele registrelor pe 6 biți în parametrii locali ai modulului. Se declară un registru \texttt{sb} care va indica dacă scrierea/citirea se va face în MSB sau în LSB pentru registrele care sunt pe 16 biți. Funcționarea modulului va avea forma unui AFD cu 3 stări (S3, S4, S5). La activarea funcției de reset toți registrii sunt trecuți în valoarea default și se revine la starea S3. Trecerea dintr-o stare în alta se realizează pe frontul crescător al ceasului. Stările automatului: \\
\\
S3: \texttt{sb} ia valoarea bitului 0 din \texttt{data\_write}. Dacă \texttt{write} este 1 se trece la starea S4, alternativ se trece la S5.\\
\\
S4: \texttt{counter\_reset} revine la valoarea 0 (acesta trebuie, conform documentației, să fie 1 doar un singur ciclu de ceas după ce este setat). Se verifică, cu un bloc \texttt{case}, adresa la care trebuie scrisă informația din \texttt{data\_write} și implicit registrul. Dacă registrul este pe 16 biți, se verifică \texttt{sb} pentru a vedea dacă \texttt{data\_write} se scrie în MSB sau LSB. Se trece în starea S3.\\
\\
S5: Se verifică, cu un bloc \texttt{case}, adresa la care se află registrul care trebuie citit în \texttt{data\_read} (dacă adresa nu este una în care se află unul dintre registre, se citește valoarea 0). Dacă registrul este pe 16 biți, se verifică \texttt{sb} pentru a vedea dacă \texttt{data\_read} se citește din MSB sau LSB. Se trece în starea S3.



\newpage
\section{COUNTER}
Se creează un counter intern și un \textit{period} intern (numit \texttt{ss}), care este un \texttt{wire} setat ca 1 pe 16 biți, shiftat la stânga cu \texttt{prescale} număr de biți. Se declară un bloc \texttt{always} activ pe frontul crescător al ceasului, care, la activarea funcției de reset, resetează \texttt{count\_val} la valoarea 0, indiferent de direcția de numărare. Registrul intern este mapat direct la ieșirea \texttt{count\_val}. Numărătoarea începe atunci când \texttt{en} are valoarea 1.\\
\\
Când \texttt{upnotdown} este 1, numărătoarea este crescătoare de la 0 la \texttt{period}. Când este 0, numărătoarea este descrescătoare; dacă valoarea curentă este 0, următorul pas va încărca valoarea \texttt{period}, altfel se decrementează.\\
\\
Incrementarea/Decrementarea lui \texttt{count\_value} se realizează atunci când counter-ul intern ajunge la valoarea \texttt{ss - 1} (deoarece începe de la 0). Când se incrementează/decrementează \texttt{count\_value}, counter-ul intern se resetează și el la 0.


\newpage


\section{PWM GENERATOR}

Se utilizează un bloc \texttt{always} activ pe frontul crescător al ceasului. La primirea semnalului de reset, valoarea lui \texttt{pwm\_out} se resetează la 0.\\
\\
Când \texttt{pwm\_en} este 0, \texttt{pwm\_out} este și el 0. Când \texttt{pwm\_en} este 1, se aplică o verificare prioritară: dacă \texttt{compare1} este egal cu \texttt{compare2}, ieșirea \texttt{pwm\_out} este forțată la 0. Altfel, se verifică în \texttt{functions} tipul de aliniere al semnalului, valoarea \texttt{count\_val} din momentul respectiv al ceasului și valoarea/valorile cu care aceasta din urmă trebuie comparată. În funcție de aceste condiții se determină semnalul \texttt{pwm\_out} la momentul dat.


\newpage
\newpage


\section{Modificări top}

Fișierul top.v a fost modificat astfel încât la apelul modulului \texttt{spi\_bridge.v}
să se utilizeze wire-rurile \texttt{byte\_sync,data\_in și data\_out, iar în apelul modulului instr\_dcd intrarea byte\_sync primește valoarea wire-ului byte\_sync din top.}


\newpage



\end{document}
